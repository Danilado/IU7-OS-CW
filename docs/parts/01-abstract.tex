\ssr{РЕФЕРАТ}

Расчетно-пояснительная записка $\text{\ztotpages}$ с., \totfig\ рис., 24 ист.

ОПЕРАЦИОННЫЕ СИСТЕМЫ, ЗАГРУЖАЕМЫЙ МОДУЛЬ ЯДРА, ПОДСИСТЕМА ВВОДА, BLUETOOTH, RFCOMM, ВИРТУАЛЬНОЕ УСТРОЙСТВО МЫШИ

Цель работы — разработка загружаемого модуля ядра Linux, обеспечивающего управление курсором мыши с использованием сенсорного экрана мобильного устройства по каналу Bluetooth.

В работе реализовано виртуальное устройство ввода типа «мышь», зарегистрированное в подсистеме input ядра Linux. Передача управляющих данных осуществляется по протоколу RFCOMM. Мобильное приложение для операционной системы Android формирует сообщения о перемещении курсора и состояниях кнопок мыши на основе событий сенсорного экрана.

Разработанный модуль обеспечивает приём сообщений по Bluetooth, их декодирование и генерацию соответствующих событий подсистемы ввода. Реализованное программное решение позволяет использовать мобильное устройство в качестве беспроводного манипулятора для управления курсором рабочей станции.

\pagebreak
