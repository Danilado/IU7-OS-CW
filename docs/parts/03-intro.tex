\ssr{ВВЕДЕНИЕ}

% Операционные системы на базе ядра Linux широко применяются на серверных, настольных и встраиваемых платформах, в том числе в составе мобильных операционных систем и специализированных дистрибутивов для рабочих станций и ноутбуков~\cite{stackoverflow-survey-2024}. Расширение функциональности ядра осуществляется с помощью загружаемых модулей, которые интегрируются в существующие подсистемы ввода-вывода и сетевого стека без пересборки ядра \cite{linux-driver-basics}.

% Цель работы --- разработка загружаемого модуля ядра Linux, реализующего виртуальное устройство мыши, и сопряжённого мобильного приложения на базе Android, обеспечивающих управление курсором рабочей станции по каналу Bluetooth с использованием сенсорного экрана телефона.

% Для достижения поставленной цели требуется решить следующие задачи:
% \begin{enumerate}
% 	\item провести анализ способов управления курсором мыши из загружаемого модуля ядра;
% 	\item провести анализ способов взаимодействия между телефоном и загружаемым модулем ядра;
% 	\item реализовать алгоритмы, загружаемый модуль ядра Linux и Android приложение для управления курсором мыши с помощью сенсорного экрана телефона.
% \end{enumerate}

Использование беспроводных устройств ввода является актуальной задачей в области операционных систем и встроенного программного обеспечения. Управление курсором с помощью мобильного устройства позволяет расширить возможности взаимодействия пользователя с рабочей станцией без применения специализированных аппаратных манипуляторов.

В данной работе разрабатывается загружаемый модуль ядра Linux, реализующий виртуальное устройство ввода типа «мышь», управление которым осуществляется с использованием сенсорного экрана мобильного телефона по каналу Bluetooth. Для формирования управляющих воздействий используется мобильное приложение под управлением операционной системы Android.

Целью работы является разработка загружаемого модуля ядра Linux и сопряжённого мобильного приложения, обеспечивающих передачу команд управления курсором от мобильного устройства к рабочей станции по беспроводному каналу связи.

Для достижения поставленной цели необходимо выполнить следующие задачи:
\begin{itemize}
	\item провести анализ способов управления курсором мыши из загружаемого модуля ядра Linux;
	\item провести анализ способов обмена данными между мобильным устройством и модулем ядра по каналу Bluetooth;
	\item разработать алгоритмы функционирования загружаемого модуля ядра и мобильного приложения;
	\item реализовать загружаемый модуль ядра Linux, обеспечивающий генерацию событий подсистемы ввода;
	\item реализовать мобильное приложение для операционной системы Android;
	\item выполнить тестирование разработанного программного комплекса.
\end{itemize}
