\ssr{ЗАКЛЮЧЕНИЕ}

В аналитическом разделе выполнен анализ способов генерации событий ввода в ядре Linux и вариантов организации обмена данными между мобильным устройством и рабочей станцией. Рассмотрены подходы к регистрации виртуального устройства в подсистеме input, использованию подсистемы uinput и интеграции через подсистему HID. В результате анализа установлено, что регистрация виртуального устройства типа «мышь» в подсистеме input обеспечивает формирование событий \texttt{EV\_REL} и \texttt{EV\_KEY} непосредственно в модуле ядра, не требует переноса логики в пользовательское пространство и не влечёт усложнения реализации за счёт поддержки HID-протокола. Данный способ выбран в качестве базового для реализации драйвера.

Также проанализированы основные варианты беспроводного обмена данными по каналу Bluetooth: прямое использование протокола L2CAP, применение профиля HID, обработка RFCOMM-соединений через устройства /dev/rfcommN в пользовательском пространстве и организация серверного RFCOMM-сокета в пространстве ядра. Показано, что использование RFCOMM-сокета непосредственно в модуле ядра обеспечивает потоковый двунаправленный канал передачи данных, позволяет передавать управляющие команды без промежуточных пользовательских прослоек и естественно сочетается с API Bluetooth операционной системы Android. На основе проведённого анализа выбран механизм обмена данными с использованием протокола RFCOMM.

На основании результатов аналитического раздела сформирована целевая архитектура программного комплекса, включающая связку «Android-приложение — RFCOMM — загружаемый модуль ядра — подсистема input». Данная архитектура положена в основу дальнейшей разработки и реализации программного решения.

В рамках работы разработаны алгоритмы функционирования загружаемого модуля ядра и мобильного приложения, реализованы приём и обработка управляющих сообщений, а также генерация событий подсистемы ввода для виртуального устройства мыши. В результате исследования подтверждена работоспособность разработанного ПО и возможность устойчивого управления курсором рабочей станции с использованием мобильного устройства по беспроводному каналу связи. Разработанное ПО полностью соответствует техническому заданию.
