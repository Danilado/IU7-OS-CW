\ssr{ЗАКЛЮЧЕНИЕ}

При выполнении курсовой работы проанализированы способы преобразования данных, полученных с сенсорного экрана смартфона в перемещение курсора мыши, способы удалённого взаимодействия ПК и смартфона и способы генерации событий ввода в ядре Linux. 

Было рассмотрено использование абсолютных координат и относительных смещений \texttt{dx/dy}. 
Был выбран способ передачи относительных смещений, так как этот подход независим от геометрии экранов, минимизирует объём данных и достаточен для перемещения курсора мыши.

Были проанализированы способы удалённого взаимодействия устройств на основе подсистемы Bluetooth.
Рассмотрено взаимодействие по протоколу L2CAP и использование RFCOMM-сокета в пространстве ядра.
Был выбран RFCOMM-сокет в пространстве ядра, так как он совместим с Android API и обеспечивает надёжный обмен без усложнения модуля.

Проанализированы способы генерации событий ввода. 
Была рассмотрена регистрация собственного \texttt{input\_dev} и вмешательство в существующий стек ввода.
Был выбран вариант с регистрацией собственного виртуального устройства ввода, так как
прямое формирование событий в модуле обеспечивает требуемый функционал без вмешательства в существующий стек ввода и не подвержено непредсказуемому воздействию внешних обработчиков.

При выполнении работы разработан загружаемый модуль ядра и Android-приложение.
Загружаемый модуль ядра выступает в качестве RFCOMM-сервера и выполняет обработку пакетов, отправленных из приложения на смартфоне. Пакеты содержат информацию об относительных смещениях, которые переводятся в события перемещения курсора мыши и события щелчков кнопок мыши.
Мобильное приложение является клиентом, подключающимся к RFCOMM-серверу модуля ядра. Оно обеспечивает считывание событие с сенсорного экрана и преобразование их в пакеты, отправляемые в загружаемый модуль ядра.

Выполнено тестирование разработанного ПО. В ходе тестирования подтверждена корректная работа модуля и приложения. 
