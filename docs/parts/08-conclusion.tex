\ssr{ЗАКЛЮЧЕНИЕ}

В ходе выполнения курсовой работы реализован загружаемый модуль ядра Linux и мобильное приложение для Android, совместно обеспечивающие управление курсором рабочей станции с помощью сенсорного экрана телефона по каналу Bluetooth.

Были решены следующие задачи, сформулированные во введении:
\begin{enumerate}
  \item проведена постановка задачи и определены требования к системе удалённого управления курсором с использованием мобильного устройства;
  \item выполнен анализ способов генерации событий мыши в контексте загружаемого модуля ядра, рассмотрены варианты с использованием подсистемы ввода, \lstinline|uinput| и драйверов HID, и обоснован выбор подхода на основе создания виртуального устройства ввода;
  \item проведён анализ вариантов обмена данными между телефоном и модулем ядра по Bluetooth, рассмотрены уровни L2CAP, RFCOMM и HID-профили, и обоснован выбор RFCOMM как потокового транспорта для передачи бинарных пакетов;
  \item разработана архитектура решения, включающая виртуальное устройство ввода в ядре и мобильное приложение, формирующее события касания;
  \item реализован загружаемый модуль ядра, регистрирующий устройство типа «мышь», принимающий данные по RFCOMM-сокету и преобразующий их в события подсистемы ввода;
  \item реализовано Android-приложение на языке Kotlin, которое устанавливает RFCOMM-соединение с рабочей станцией и отправляет пакеты, формируемые на основе движений пальца и нажатий виртуальных кнопок;
  \item проведена экспериментальная проверка работоспособности системы и демонстрация управления курсором и кнопками мыши с мобильного устройства.
\end{enumerate}

Поставленная цель, заключавшаяся в реализации драйвера мыши для управления курсором с помощью сенсорного экрана телефона, достигнута. Созданный программный комплекс интегрируется с подсистемой ввода Linux и стеками Bluetooth рабочих станций и мобильных устройств, обеспечивая функциональность, соответствующую требованиям технического задания.
