\subsection{Анализ способов преобразования касаний сенсорного экрана в перемещение курсора мыши}

При интеграции мобильного приложения, формирующего события касаний, и загружаемого модуля ядра, эмулирующего поведение мыши, необходимо определить способ сопоставления данных сенсорного экрана с координатами курсора.

В соответствии с логикой построения указательных устройств рассмотрены два подхода:
\begin{enumerate}
\item преобразование координат касаний в \textit{абсолютные} координаты на экране ПК;
\item преобразование перемещения пальца в \textit{относительные} смещения курсора.
\end{enumerate}

Абсолютные координаты требуют точного соответствия геометрии сенсорного устройства и рабочего стола, а также нормализации значений, поступающих от Android-приложения, к диапазону координат подсистемы ввода ядра Linux. Такой способ приводит к ряду ограничений: несовпадение пропорций сенсорной панели и монитора, масштабирование рабочего стола и наличие нескольких экранов приводят к неоднозначности отображения координат. 

Абсолютное позиционирование предполагает передачу полного набора координат при каждом событии касания, что повышает интенсивность обмена данными по Bluetooth и увеличивает требования к стабильности канала связи.

Использование относительных координат основано на передаче смещений \textit{dx, dy}, вычисляемых из последовательности касаний (или непрерывного перемещения пальца). 
Такой подход соответствует модели классических компьютерных мышей, где устройство генерирует относительные изменения положения независимо от абсолютной позиции курсора. 

Кроме того, относительные смещения естественным образом интегрируются в архитектуру Android-приложения, где компонент обработчика касаний оперирует разностями координат, что исключает необходимость приведения значений к масштабу удалённого экрана.

С учётом перечисленных факторов, был выбран способ передачи \textit{относительных координат}. 
Он достаточен для обеспечения перемещения курсора мыши и не требует конфигурации для сопоставления масштабов экрана.
