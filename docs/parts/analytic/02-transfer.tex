\subsection{Анализ способов обмена данными между смартфоном и модулем ядра}

Передача данных между мобильным устройством и загружаемым модулем ядра может быть реализована различными способами, включая сетевые протоколы и беспроводные стеки передачи данных.

Для решения поставленной задачи выбрано беспроводное соединение на основе подсистемы Bluetooth. Оно обеспечивает устойчивый двунаправленный канал связи, нативно поддерживается в ядре Linux через подсистему \texttt{BlueZ} и предоставляет API, совместимый с типовыми средствами разработки Android-приложений.

Стек Bluetooth в Linux реализован как подсистема \texttt{BlueZ}, включающая поддержку протокола L2CAP, протокола RFCOMM и вспомогательных служб~\cite{bluetooth-core-spec,bluez-l2cap-wiki,bluez-rfcomm-wiki}. На уровне ядра для взаимодействия с Bluetooth-устройствами используются сокеты семейства \texttt{PF\_BLUETOOTH} с такими протоколами, как \texttt{BTPROTO\_L2CAP} и \texttt{BTPROTO\_RFCOMM}~\cite{bluetooth-core-spec,kernel-net-kapi}. В пространстве ядра доступен интерфейс для создания и использования таких сокетов, обеспечивающий работу с Bluetooth-соединениями~\cite{kernel-net-kapi}.

С точки зрения обмена данными между смартфоном и модулем ядра возможны следующие варианты:
\begin{enumerate}
	\item использование RFCOMM-сокетов в пространстве ядра;
	\item работа непосредственно с L2CAP в пространстве ядра.
\end{enumerate}

\

\textit{RFCOMM-сокеты в пространстве ядра}

Протокол RFCOMM реализует поверх L2CAP байтовый поток, логически аналогичный последовательному порту, и применяется для построения сервисов, требующих надёжного двунаправленного канала~\cite{bluetooth-core-spec,bluez-rfcomm-wiki}. В ядре Linux поддержка RFCOMM интегрирована в сетевой стек, что позволяет создавать серверные и клиентские сокеты с использованием семейства \texttt{PF\_BLUETOOTH} и протокола \texttt{BTPROTO\_RFCOMM}~\cite{bluetooth-core-spec,linux-rfcomm-core}. Ядро Linux предоставляет структуры и функции для работы с RFCOMM, объявленные в заголовках \texttt{<net/bluetooth/bluetooth.h>} и \texttt{<net/bluetooth/rfcomm.h>}. Адресация RFCOMM-сокета выполняется с использованием структуры адреса, содержащей семейство, Bluetooth-адрес удалённого устройства и номер канала.

На стороне Android-приложения подключение к такому сервису выполняется через API класса \texttt{BluetoothSocket}, который инкапсулирует установление RFCOMM-соединения по указанному UUID сервиса~\cite{android-bt-connect,BluetoothSocket-doc}. Таким образом формируется связка: серверный RFCOMM-сокет в пространстве ядра и клиентское соединение в приложении Android.

В рамках данной разработки загружаемый модуль ядра создаёт серверный RFCOMM-сокет и принимает входящие соединения, а мобильное приложение выступает клиентом этого сервера.

При использовании RFCOMM модуль ядра получает поток байтов непосредственно от смартфона, что позволяет задать прикладной протокол управления курсором (например, фиксированный формат кадров с координатами и битовой маской кнопок) без вовлечения дополнительных уровней абстракции~\cite{bluetooth-core-spec,bluez-rfcomm-wiki}. Обработка входящего потока выполняется в  обработчиках на стороне загружаемого модуля ядра, что упрощает синхронизацию с подсистемой ввода~\cite{kernel-net-kapi}.

\

\textit{Протокол L2CAP}

L2CAP представляет собой базовый протокол Bluetooth, обеспечивающий мультиплексирование каналов и передачу пакетов между устройствами~\cite{bluetooth-core-spec,bluez-l2cap-wiki}. Реализация L2CAP входит в стек BlueZ ядра Linux; интерфейс для работы с L2CAP предоставляется ядром через функции и структуры, объявленные в заголовке \texttt{<net/bluetooth/l2cap.h>}~\cite{bluetooth-core-spec,l2cap-man}. Прямое использование L2CAP даёт доступ к более низкому уровню стека и предоставляет гибкость при реализации собственных протоколов, но требует дополнительной обработки параметров канала и управления MTU~\cite{l2cap-man}.

В контексте рассматриваемой задачи использование L2CAP в качестве средства передачи данных для прикладного протокола управления курсором приводит к усложнению логики модуля ядра и дублированию функциональности, уже реализованной в RFCOMM, как надстройке над L2CAP~\cite{bluetooth-core-spec,bluez-l2cap-wiki}.
Кроме того, на стороне Android типовые высокоуровневые API ориентированы на RFCOMM-сервисы, что делает прямую работу с L2CAP при реализации мобильного приложения неоправданным ограничением~\cite{android-bt-connect}.

% В результате анализа способов передачи данных в модуль ядра по протоколу Bluetooth, получены следующие выводы:
% \begin{itemize}
% 	\item L2CAP предоставляет универсальный транспортный уровень и используется в качестве основы для протоколов более высокого уровня, однако при прямом использовании влечёт усложнение модуля за счёт необходимости дополнительной обработки параметров канала~\cite{bluetooth-core-spec,bluez-l2cap-wiki,l2cap-man};
% 	\item RFCOMM-сокет в пространстве ядра обеспечивает потоковый канал непосредственно в модуль, использует реализованный протокол поверх L2CAP и поддерживает соединение с высокоуровневым API Android-приложения~\cite{linux-rfcomm-core,kernel-net-kapi,android-bt-connect,BluetoothSocket-doc}.
% \end{itemize}
