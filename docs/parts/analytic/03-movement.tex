\subsection{Способы управления курсором мыши из загружаемого модуля ядра}

% Подсистема ввода ядра Linux представляет абстракцию устройств ввода в виде структуры \texttt{struct input\_dev}. Драйверы регистрируют такие устройства в ядре, после чего события от них транслируются в стандартные интерфейсы \texttt{/dev/input} и, далее, в графические подсистемы и оконные менеджеры~\cite{linux-input-docs,input-programming}. 

Для управления курсором из загружаемого модуля рассматриваются три класса подходов:
\begin{enumerate}
	\item регистрация виртуального устройства ввода в подсистеме input;
	\item генерация событий через подсистему \texttt{uinput} из пользовательского пространства;
	\item интеграция с подсистемой HID с эмуляцией HID-манипулятора.
\end{enumerate}

\

\textit{Виртуальное устройство в подсистеме input}

Базовый способ генерации событий мыши из загружаемого модуля ядра заключается в регистрации виртуального устройства ввода через подсистему input~\cite{linux-input-docs,input-programming}. Драйвер выделяет и инициализирует структуру \texttt{struct input\_dev}, заполняя сведения об идентификаторе устройства и поддерживаемых типах событий, а затем регистрирует её в input-core, после чего в системе появляется соответствующее устройство мыши~\cite{linux-input-docs}. Для устройства мыши обычно указываются типы событий \texttt{EV\_REL} (относительное перемещение по осям \texttt{REL\_X}, \texttt{REL\_Y}) и \texttt{EV\_KEY} (нажатия \texttt{BTN\_LEFT}, \texttt{BTN\_RIGHT})~\cite{linux-input-docs,linux-hid-docs,hidintro}.

Генерация событий для зарегистрированного устройства выполняется специализированными функциями подсистемы ввода, такими как \texttt{input\_report\_rel}, \texttt{input\_report\_key} и \texttt{input\_sync}, которые обновляют внутреннее состояние устройства и доставляют события всем подписанным обработчикам~\cite{linux-input-docs,input-programming}. В контексте разрабатываемого драйвера каждое поступившее сообщение от смартфона интерпретируется как набор элементарных действий мыши (смещение по осям и изменение состояний кнопок) и преобразуется в одну или несколько последовательностей вызовов указанных функций. Такой подход обеспечивает интеграцию с архитектурой ввода и делает виртуальное устройство неотличимым от аппаратной мыши для остального программного обеспечения~\cite{linux-input-docs,linux-hid-docs}.

\

\textit{Подсистема \texorpdfstring{\texttt{uinput}}{uinput} и генерация событий из пользовательского пространства}

Подсистема \texttt{uinput} реализует интерфейс, позволяющий пользовательским процессам создавать виртуальные устройства ввода и генерировать события от их имени через специальное символьное устройство \texttt{/dev/uinput}~\cite{uinput-docs,linux-input-docs}. Пользовательское приложение описывает возможности устройства, после чего отправляет структуры \texttt{struct input\_event}, которые ядро интерпретирует как события подсистемы ввода~\cite{uinput-docs,input-programming}. Этот механизм применяется для реализации эмуляторов устройств ввода и прикладных программ, инжектирующих события в пользовательском пространстве.

Для решения рассматриваемой задачи применение \texttt{uinput} означало бы перенос логики приёма данных по Bluetooth и обработки протокола с модуля ядра в пользовательское приложение, которое затем транслировало бы события мыши через \texttt{/dev/uinput}. При таком подходе загружаемый модуль ядра фактически не участвовал бы в формировании событий, а являлся бы вспомогательным компонентом либо полностью отсутствовал. Этот подход не совпадает с утверждённым заданием на курсовую работу.

\

\textit{Интеграция с подсистемой HID и интерфейс UHID}

Подсистема HID ядра Linux обеспечивает поддержку устройств ввода, реализующих стандартный HID-протокол (клавиатуры, мыши, графические планшеты и др.), и опирается на разделение на транспортные драйверы и HID-core~\cite{linux-hid-docs,hidintro,hid-transport}. Транспортный драйвер отвечает за доставку HID-отчётов от конкретной шины (USB, Bluetooth и т.п.), а HID-core интерпретирует отчёты, формируя события подсистемы ввода~\cite{linux-hid-docs,hiddev-doc}. Интерфейс UHID предоставляет возможность создавать HID-устройства из пользовательского пространства: пользовательский процесс взаимодействует с устройством \texttt{/dev/uhid}, отправляя события, которые далее обрабатываются HID-core и преобразуются в события ввода~\cite{uhid-doc,linux-hid-docs}.

Использование HID-подсистемы для эмуляции мыши обеспечивает совместимость с существующим стеком драйверов и поддержкой HID в ядре~\cite{linux-hid-docs,hidintro}. Однако для учебной задачи, в которой акцент сделан на реализации логики непосредственно в модуле ядра, такой подход требует разработки HID-дескриптора, генерации корректных HID-отчётов и учёта дополнительных особенностей HID-профиля, что усложняет структуру драйвера без необходимости использования расширенных возможностей HID-протокола~\cite{hid-transport,hiddev-doc}.

\

\textit{Другие способы вмешательства в стек ввода}

На практике применяются подходы, основанные на перехвате или модификации событий на поздних этапах стека ввода, например через перехват операций устройств \lstinline|/dev/input/eventX| или вмешательство в обработчики графической подсистемы~\cite{linux-input-docs}. Такие решения опираются на модификацию существующих драйверов либо на установку промежуточных слоёв между ядром и пользовательскими приложениями. Для курсовой работы по разработке загружаемого модуля ядра эти варианты не соответствуют постановке задачи, так как не создают отдельного драйвера мыши, а изменяют поведение уже существующих компонентов.

\

\textit{Обоснование выбора подсистемы input и виртуального устройства мыши}

В результате анализа рассмотренных подходов сформулированы следующие выводы:
\begin{itemize}
	\item регистрация виртуального устройства через подсистему input обеспечивает интеграцию с ядром, контролируемый набор зависимостей и представление мыши в системе как стандартного устройства ввода, управляемого загружаемым модулем ядра~\cite{linux-input-docs,input-programming};
	\item использование \texttt{uinput} переносит ключевую логику в пользовательское пространство и приводит к тому, что основная часть функциональности оказывается реализованной вне модуля ядра~\cite{linux-driver-basics,uinput-docs};
	\item интеграция через HID и UHID требует разработки HID-дескриптора и обработки HID-отчётов, что не является необходимым для эмуляции мыши с фиксированным набором событий, но усложняет реализацию драйвера~\cite{linux-hid-docs,hidintro}.
\end{itemize}

В связи с этим, в разрабатываемой системе выбран подход, основанный на регистрации виртуального устройства мыши в подсистеме input и генерации событий с помощью функций подсистемы ввода из загружаемого модуля ядра~\cite{linux-input-docs,input-programming}.
Этот способ обладает достаточным функционалом для выполнения работы и совпадает с утверждённым заданием.
